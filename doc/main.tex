\documentclass{ecai}
\usepackage{times}
\usepackage{graphicx}
\usepackage{latexsym}
\usepackage[utf8]{inputenc}
\usepackage{graphicx}
\usepackage{xspace}
\usepackage{amsmath}
\usepackage{booktabs,multirow}
\usepackage{verbatim}
\usepackage{graphics}
\usepackage{hyperref}
\newcommand{\citet}{\cite}

\newcommand{\TODO}[1]{\textcolor[rgb]{1.00,0.00,0.00}{todo: #1} }
\newcommand{\VALE}[1]{\textcolor[rgb]{0.00,0.00,0.54296875}{vale: #1} }


\newcommand\bigforall{\mbox{\Large $\mathsurround0pt\forall$}}
\newcommand\bigexists{\mbox{\Large $\mathsurround0pt\exists$}}

\usepackage{acronym}
\acrodef{GTTT}{Generalized Tic-tac-toe}
\acrodef{MCTS}{Monte Carlo Tree Search}
\acrodef{PNS}{Proof Number Search}
\acrodef{QBF}{Quantified Boolean Formula}
%\acrodefplural{QBF}[QBF]{Quantified Boolean Formulae}
\acrodefplural{QBF}{Quantified Boolean Formulae}

\newcommand{\white}{\textsc{W} }
\newcommand{\black}{\textsc{B} }
\newcommand{\board}{{\mathsf{board}}}
\newcommand{\timee}{\mathsf{time}}
\newcommand{\lose}{\mathsf{lose}}
\newcommand{\win}{\mathsf{win}}
\newcommand{\move}{\mathsf{move}}
\newcommand{\occupied}{\mathsf{occupied}}
\newcommand{\ld}{\mathsf{ladder}}
\newcommand{\cnt}{\mathsf{count}}
\newcommand{\moveL}{\mathsf{moveL}}
\newcommand{\depth}{\mathsf{F}}
\newcommand{\size}{\mathsf{N}}
\newcommand{\ENC}{\emph{COR}\xspace}

\begin{document}

\title{BULE: A Simple Grounding Language for SAT}
\author{Valentin Mayer-Eichberger\institute{Technische Universit\"at Berlin, Germany}}
\maketitle

\begin{abstract}
CNF formulas are usually generated by a problem specific algorithm, translated
    from a more expressive language such as constraint programming or specified
    in a fragment of first order logic. When encodings are  presented in the
    literature the inner workings are often hidden in subscripts of variables
    and hard to decipher.

In this paper we propose the grounding language \emph{Bule} that is easy to
    write and read, and helps to generate, reason and compare encodings. We aim
    to simplify the approaches above and remove just enough layers between
    creating and implementing an encodings to facilitate prototyping and
    debugging. In this paper we formally define the language \emph{Bule}, show
    its application through a case study and explain how to use the implemented
    grounder. 
\end{abstract}

\section{Introduction}
\section{Definition}
\section{Conclusion}

This is the counter encoding found in \cite{Sinz2005}.

We thank Cameron Browne, Ryan Hayward, and Bjarne Toft for providing us with
the hand-crafted Hex puzzles. We also thank Mikoláš Janota for his feedback on
earlier versions of the encoding.

\bibliographystyle{ecai}
\bibliography{main}

\end{document}
